% Options for packages loaded elsewhere
\PassOptionsToPackage{unicode}{hyperref}
\PassOptionsToPackage{hyphens}{url}
%
\documentclass[
]{article}
\usepackage{amsmath,amssymb}
\usepackage{iftex}
\ifPDFTeX
  \usepackage[T1]{fontenc}
  \usepackage[utf8]{inputenc}
  \usepackage{textcomp} % provide euro and other symbols
\else % if luatex or xetex
  \usepackage{unicode-math} % this also loads fontspec
  \defaultfontfeatures{Scale=MatchLowercase}
  \defaultfontfeatures[\rmfamily]{Ligatures=TeX,Scale=1}
\fi
\usepackage{lmodern}
\ifPDFTeX\else
  % xetex/luatex font selection
\fi
% Use upquote if available, for straight quotes in verbatim environments
\IfFileExists{upquote.sty}{\usepackage{upquote}}{}
\IfFileExists{microtype.sty}{% use microtype if available
  \usepackage[]{microtype}
  \UseMicrotypeSet[protrusion]{basicmath} % disable protrusion for tt fonts
}{}
\makeatletter
\@ifundefined{KOMAClassName}{% if non-KOMA class
  \IfFileExists{parskip.sty}{%
    \usepackage{parskip}
  }{% else
    \setlength{\parindent}{0pt}
    \setlength{\parskip}{6pt plus 2pt minus 1pt}}
}{% if KOMA class
  \KOMAoptions{parskip=half}}
\makeatother
\usepackage{xcolor}
\usepackage[margin=1in]{geometry}
\usepackage{color}
\usepackage{fancyvrb}
\newcommand{\VerbBar}{|}
\newcommand{\VERB}{\Verb[commandchars=\\\{\}]}
\DefineVerbatimEnvironment{Highlighting}{Verbatim}{commandchars=\\\{\}}
% Add ',fontsize=\small' for more characters per line
\usepackage{framed}
\definecolor{shadecolor}{RGB}{248,248,248}
\newenvironment{Shaded}{\begin{snugshade}}{\end{snugshade}}
\newcommand{\AlertTok}[1]{\textcolor[rgb]{0.94,0.16,0.16}{#1}}
\newcommand{\AnnotationTok}[1]{\textcolor[rgb]{0.56,0.35,0.01}{\textbf{\textit{#1}}}}
\newcommand{\AttributeTok}[1]{\textcolor[rgb]{0.13,0.29,0.53}{#1}}
\newcommand{\BaseNTok}[1]{\textcolor[rgb]{0.00,0.00,0.81}{#1}}
\newcommand{\BuiltInTok}[1]{#1}
\newcommand{\CharTok}[1]{\textcolor[rgb]{0.31,0.60,0.02}{#1}}
\newcommand{\CommentTok}[1]{\textcolor[rgb]{0.56,0.35,0.01}{\textit{#1}}}
\newcommand{\CommentVarTok}[1]{\textcolor[rgb]{0.56,0.35,0.01}{\textbf{\textit{#1}}}}
\newcommand{\ConstantTok}[1]{\textcolor[rgb]{0.56,0.35,0.01}{#1}}
\newcommand{\ControlFlowTok}[1]{\textcolor[rgb]{0.13,0.29,0.53}{\textbf{#1}}}
\newcommand{\DataTypeTok}[1]{\textcolor[rgb]{0.13,0.29,0.53}{#1}}
\newcommand{\DecValTok}[1]{\textcolor[rgb]{0.00,0.00,0.81}{#1}}
\newcommand{\DocumentationTok}[1]{\textcolor[rgb]{0.56,0.35,0.01}{\textbf{\textit{#1}}}}
\newcommand{\ErrorTok}[1]{\textcolor[rgb]{0.64,0.00,0.00}{\textbf{#1}}}
\newcommand{\ExtensionTok}[1]{#1}
\newcommand{\FloatTok}[1]{\textcolor[rgb]{0.00,0.00,0.81}{#1}}
\newcommand{\FunctionTok}[1]{\textcolor[rgb]{0.13,0.29,0.53}{\textbf{#1}}}
\newcommand{\ImportTok}[1]{#1}
\newcommand{\InformationTok}[1]{\textcolor[rgb]{0.56,0.35,0.01}{\textbf{\textit{#1}}}}
\newcommand{\KeywordTok}[1]{\textcolor[rgb]{0.13,0.29,0.53}{\textbf{#1}}}
\newcommand{\NormalTok}[1]{#1}
\newcommand{\OperatorTok}[1]{\textcolor[rgb]{0.81,0.36,0.00}{\textbf{#1}}}
\newcommand{\OtherTok}[1]{\textcolor[rgb]{0.56,0.35,0.01}{#1}}
\newcommand{\PreprocessorTok}[1]{\textcolor[rgb]{0.56,0.35,0.01}{\textit{#1}}}
\newcommand{\RegionMarkerTok}[1]{#1}
\newcommand{\SpecialCharTok}[1]{\textcolor[rgb]{0.81,0.36,0.00}{\textbf{#1}}}
\newcommand{\SpecialStringTok}[1]{\textcolor[rgb]{0.31,0.60,0.02}{#1}}
\newcommand{\StringTok}[1]{\textcolor[rgb]{0.31,0.60,0.02}{#1}}
\newcommand{\VariableTok}[1]{\textcolor[rgb]{0.00,0.00,0.00}{#1}}
\newcommand{\VerbatimStringTok}[1]{\textcolor[rgb]{0.31,0.60,0.02}{#1}}
\newcommand{\WarningTok}[1]{\textcolor[rgb]{0.56,0.35,0.01}{\textbf{\textit{#1}}}}
\usepackage{graphicx}
\makeatletter
\def\maxwidth{\ifdim\Gin@nat@width>\linewidth\linewidth\else\Gin@nat@width\fi}
\def\maxheight{\ifdim\Gin@nat@height>\textheight\textheight\else\Gin@nat@height\fi}
\makeatother
% Scale images if necessary, so that they will not overflow the page
% margins by default, and it is still possible to overwrite the defaults
% using explicit options in \includegraphics[width, height, ...]{}
\setkeys{Gin}{width=\maxwidth,height=\maxheight,keepaspectratio}
% Set default figure placement to htbp
\makeatletter
\def\fps@figure{htbp}
\makeatother
\setlength{\emergencystretch}{3em} % prevent overfull lines
\providecommand{\tightlist}{%
  \setlength{\itemsep}{0pt}\setlength{\parskip}{0pt}}
\setcounter{secnumdepth}{-\maxdimen} % remove section numbering
\ifLuaTeX
  \usepackage{selnolig}  % disable illegal ligatures
\fi
\usepackage{bookmark}
\IfFileExists{xurl.sty}{\usepackage{xurl}}{} % add URL line breaks if available
\urlstyle{same}
\hypersetup{
  pdftitle={Write-up},
  hidelinks,
  pdfcreator={LaTeX via pandoc}}

\title{Write-up}
\author{}
\date{\vspace{-2.5em}2024-12-06}

\begin{document}
\maketitle

\begin{Shaded}
\begin{Highlighting}[]
\FunctionTok{setwd}\NormalTok{(}\StringTok{"/Users/Lenovo/Documents/GitHub/DAP2{-}final{-}project{-}andiyoga34"}\NormalTok{) }\CommentTok{\#set working directory as place for knitting output in PDF}
\end{Highlighting}
\end{Shaded}

The Trajectory of Countries' Economy and Fiscal During \& Post Pandemic
Covid-19

Research Question

\begin{enumerate}
\def\labelenumi{\arabic{enumi}.}
\item
  How are the economy and fiscal of developed and developing countries
  impacted by pandemic Covid-19 and how fast they recover?
\item
  Which one is more influential on countries' debt performance: economy
  or revenue performance?
\end{enumerate}

My approach for the first research question is by using descriptive
analysis and text analysis. For descriptive analysis, the inputs are
four datasets I gathered from World Bank Database (i.e.~GDP, GDP Per
Capita, Debt, and Revenue) that I later merged to show the trajectory of
countries' economy and fiscal (indicated by debt and revenue) during and
post-Covid19. To distinguish developed and developing countries in this
analysis I use GPD Per Capita data and threshold used by the World Bank
to categorize High Income Countries: 12,000 USD. To see quickly how the
economy of countries immediately impact by the pandemic in 2020, I also
created choropleth of World Heat Map displaying annual GDP growth of
each country sourced from IMF data.

For the text analysis, I used World Bank's Global Economic Prospect
Report (January 2023) that covered how the countries started recover
from the pandemic in 2022 and its trajectory going forward.

For the second research question, my approach is using regression
models. First regression is Debt on GDP to see the relationship between
the two, and my second regression is Debt on Revenue. Both regression
models deploy three variables of control i.e.~GDP Per Capita (to control
for countries stage of development), countries fixed effect (to control
for countries different characteristic outside stage of development
e.g.~institutional quality and geographic location), and time fixed
effect (accounting for common shocks i.e.~pandemic year affecting all
countries at time 𝑡).

What the all R files do with the coding: 1. data.R : doing all data
wrangling to create one dataset as a base for my descriptive analysis
and regression models

\begin{enumerate}
\def\labelenumi{\arabic{enumi}.}
\setcounter{enumi}{1}
\item
  models.R: creating regression models and run them.
\item
  textprocess.R: loading the pre-downloaded WB's GEP January 2023 Report
  and parsed it to text and tokenize then to be processed further
  including using AFINN lexicon
\item
  staticplots.R : making all the static plots in the second tab (World
  Data including Developing and Developed countries), third tab
  (choropleth of World Heat map of annual GDP growth in 2020), fourth
  tab (regressions plots), and fifth tab (text analyses plots) in my
  shinyapp dashboard:
\item
  shinyapp.R: running the code of shinyapp to display my dashboard
  displaying interactive graphs of individual country's debt, revenue,
  and GDP in the first tabs and all static plots mentioned above created
  by staticplots.R
\end{enumerate}

The weaknesses and difficulties of the project:

\begin{enumerate}
\def\labelenumi{\arabic{enumi}.}
\item
  There are so many NAs in the debt data from the World Bank database. I
  don't know why that's the case, but when I tried to compare it with
  IMF data, the number of NAs are pretty much the same. The existence of
  so many NAs here might affect the results of the regressions.
\item
  Difficulty in merging all countries successfully from the IMF data to
  the World Data (rnaturalearth library) due to differences of wording
  and spelling for certain countries.
\end{enumerate}

Discussion of the Results and Next Steps:

\begin{enumerate}
\def\labelenumi{\arabic{enumi}.}
\item
  Developed countries cope way better than the developing countries
  during the pandemic and recover much better post-pandemic.
\item
  In this analysis, both revenue and GDP movement seem to have weak,
  positive correlations with debt performance and are statistically
  insignificant in affecting the debt trajectory. However, in addition
  to the common caveat of ``correlation is not causation'', the
  limitation of the data (so many NAs in the debt data) should make us
  take a careful conclusion on this result.
\item
  In the next research/analysis, it might be useful to see the
  projection of the debt, GDP, and revenue in some years in the future
  using some forecasting filters learned in class (e.g.~HP filter).
\end{enumerate}

\end{document}
